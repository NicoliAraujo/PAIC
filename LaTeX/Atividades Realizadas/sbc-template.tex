\documentclass[12pt]{article}
\usepackage{pgf,tikz}
\usetikzlibrary{arrows}
\usepackage{sbc-template}
\usepackage{graphicx,url}
\usepackage{courier}
\usepackage[latin1]{inputenc}
\usepackage[brazil]{babel}
\usepackage{amssymb}
\usepackage{amsmath}
\usepackage{algorithmic}
\usepackage{float, cite}
\usepackage{booktabs}
\usepackage[bf,sf,footnotesize,indent,justification=centering]{caption}
\usepackage{framed}
\usepackage[caption=true,font=footnotesize]{subfig}
\usepackage{relsize}
\newcommand\Cpp{C\nolinebreak[4]\hspace{-.05em}\raisebox{.4ex}{\relsize{-3}{\textbf{++}}}\ }
\captionsetup[subfigure]{justification=centering,labelfont={bf,sf},textfont={bf,sf,footnotesize},singlelinecheck=off,justification=centering}
\captionsetup[figure]{justification=centering,labelfont={bf,sf},textfont={bf,sf,footnotesize},singlelinecheck=off}
\captionsetup[table]{justification=centering,labelfont={bf,sf},textfont={bf,sf,footnotesize},singlelinecheck=off,justification=raggedright}

\newcounter{definitionCounter}
\newenvironment{definition}[1]{
    \refstepcounter{definitionCounter}
\vspace{0.35cm}
\noindent \textbf{Defini��o \thedefinitionCounter} (\textbf{#1}).
}{\vspace{0.35cm}}


\usepackage{color}
\usepackage{listings}
\lstset{language=C++,
                basicstyle=\ttfamily,
                keywordstyle=\color{blue}\ttfamily,
                stringstyle=\color{red}\ttfamily,
                commentstyle=\color{green}\ttfamily,
                morecomment=[l][\color{magenta}]{\#},
                numbers=left,
                stepnumber=1,
                firstnumber=1,
                numberfirstline=true,
}

\usepackage{tikz}
\usetikzlibrary{arrows}





\hyphenation{}


\sloppy

\title{Aproxima��o dos valores de $\pi$\\no Raspberry Pi}

\author{Janderson Lira, Ello� B. Guedes}

\address{N�cleo de Computa��o\\
Escola Superior de Tecnologia\\
Universidade do Estado do Amazonas\\
Av. Darcy Vargas, 1200 -- Manaus -- Amazonas
    \email{jnl.eng@uea.edu.br, ebgcosta@uea.edu.br}
}



\begin{document}
\selectlanguage{brazil}

\maketitle

\section{Atividades desenvolvidas}
\begin{tabular}{|c|c|c|c|c|}
\hline
\textbf{N�MERO $\pi$} & \textbf{Pesquisar} & \textbf{Estudar} & \textbf{Implementar} & \textbf{LaTeX}\\
\hline
Hist�ria de $\pi$ - A.C & ok & ok & ok & ok\\
\hline
Hist�ria de $\pi$ - D.C & ok & ok &  & \\
\hline
M�t. Iterativos de $\pi$ & ok & ok & ok & ---\\
\hline
M�t. de Monte Carlo & ok &  &  & ---\\
\hline
\textbf{RASPBERRY PI} & \textbf{Pesquisar} & \textbf{Estudar} & \textbf{Implementar} & \textbf{LaTeX}\\
\hline
Raspberry Pi - Hardware & ok & ok & ok & ok\\
\hline
Raspberry Pi - Software & ok & ok &  & \\
\hline
Aplica��es do Raspberry Pi & ok & ok & ok & ok\\
\hline
Testar M�t. Iterativos & ok & ok & Fazendo & ---\\
\hline
\textbf{M�T. ITERATIVOS EM PYTHON} & \textbf{Pesquisar} & \textbf{Estudar} & \textbf{Implementar} & \textbf{LaTeX}\\
\hline
M�t. Iterativos - Leibniz & ok & ok & ok & ok\\
\hline
M�t. Iterativos - Euler & ok & ok & ok & ok\\
\hline
M�t. Iterativos - Chudnovsky & ok & ok & ok & \\
\hline
M�t. Iterativos - Wallis & ok & ok & ok & \\
\hline
M�t. Iterativos - Ramanujan & ok & ok & ok & \\
\hline
\textbf{OUTRAS COISAS} & \textbf{Pesquisar} & \textbf{Estudar} & \textbf{Implementar} & \textbf{LaTeX}\\
\hline
Script para gera��o de gr�ficos & ok & ok & ok & ---\\
\hline
Passar c�digo para Eclipse & ok & ok & Refazer & ---\\
\hline
Padr�es de Projetos - Facade & ok & ok & Refazer & ---\\
\hline
Utiliza��o do GitHub & Fazendo & Fazendo & & ---\\
\hline
Biblioteca \textit{mpmath} & Fazendo & Fazendo & & ---\\
\hline
\end{tabular}

\end{document}
