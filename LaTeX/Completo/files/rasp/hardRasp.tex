O \textit{Raspberry Pi}(Rasp) � um computador de pequeno porte, com consider�vel poder de processamento e encapsulado em uma �nica placa de circuito impresso. A vers�o utilizado nesse projeto � o \textit{Rasberry Pi B}(RaspB), que possui dimens�es de 85,6x53,98x17mm.

Baseado na arquitetura ARM11, o processador do RaspB � um \textit{System-on-Chip}(SoC) Broadcom BCM2835. Ao BCM2835, podem ser acoplados:
\begin{itemize}
\item{Temporizadores};
\item{Controladores de interrup��es};
\item{\textit{General Purpose Input/Output}(GPIO)};
\item{\textit{Universal Serial Bus}(USB)};
\item{Controlador de \textit{Direct Memory Access}(DMA)};
\item{entre outros}.
\end{itemize}
O RaspB possui uma CPU ARM1176JZFS com clock de 700MHz(isso significa que o processador do RaspB pode realizar 700 milh�es de opera��es por segundo), acompanhado de uma mem�ria principal SDRAM de 512MB. Sua GPU possui um co-processador multim�dia VideoCore IV$^\circledR$ Dual Core e um acelerador gr�fico OpenVG$^\circledR$ Open GL ES 2.0 com performace de at� 24GFLOPS, capas de reproduzir at� 1080p de resolu��o \cite{Holton:RaspArch}.

O boot do sistema operacional � dado atrav�s de cart�o \textit{microSD} inserido no RaspB. Este modelo � apenas compat�vel com distribui��es Linux. A fabricante aconselha que, para melhor utiliza��o dos recursos do RaspB, seja usada a distribui��o \textit{Raspbian Wheezy}  \cite{Upton:UserGuide}.

Para melhor funcionamento do RaspB, � necess�ria uma fonte microUSB que possa fornecer 5V de tens�o e uma corrente entre 700 e 1200mA. Com corrente abaixo dessa faixa, o RaspB n�o funcionar� corretamente e, correntes acima dessa faixa, podem causar s�rios danos ao RaspB. Sobre as conex�es de Entrada/Sa�da(I/O, do ingl�s \emph{Input/Outpu}), o RaspB possui:
\begin{itemize}
\item{Duas portas USB 2.0};
\item{Duas sa�das de v�deo e �udio, sendo uma \emph{High-Definition Multimedia Interface}(HDMI) e outra RCA de 3,5mm};
\item{Interface serial para c�mera  MIPI(CSI-2)};
\item{Interface serial para display(DSI) com conector para cabo \emph{flat}}.
\end{itemize}