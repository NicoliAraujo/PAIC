Nascido na cidade de Leipzig, em 1646, Gottfried Wilhelm von Leibniz era fil�sofo, matem�tico e conselheiro pol�tico, desenvolveu importantes teorias em suas �reas de atua��o e � conhecido, ao lado de Isaac Newton, como um dos pais do c�lculo moderno, devido suas descobertas sobre o c�lculo diferencial. Leibniz faleceu na cidade de Hanover, em 1716 \cite{Look:GottfriedLeibniz}.

Dentre as contribui��es de Leibniz para a �rea da matem�tica, est� a f�rmula para aproxima��o de $\pi$, obtida por volta de 1670. Existe uma grande semelhan�a a f�rmula de Leibniz e a do matem�tico escoc�s James Gregory (1638-1675), e portanto, ficou conhecido como f�rmula de Gregoryl-Leibniz \cite{Wendpap:AbordagemPi}.

\begin{equation}
\pi = 4(\frac{1}{1}-\frac{1}{3}+\frac{1}{5}-\frac{1}{7}+\ldots) =4\sum_{n=0}^{\infty}\frac{(-1)^n}{2n + 1}\label{form:Leibniz}
\end{equation}

\definecolor{ca}{rgb}{ 0.0 , 0.0 , 0.8 }
\definecolor{cb}{rgb}{ 0.0 , 0.0 , 1.0 }
\begin{tikzpicture}[line cap=round,line join=round,>=triangle 45,x=1.0cm,y=1.0cm]
\draw[->,color=black] (-2,0) -- (12,0);
\foreach \x in {-1,1,2,3,4,5,6,7,8,9,10}
\draw[shift={(\x,0)},color=black] (0pt,1pt) -- (0pt,-1pt) node[below] {\footnotesize $\x$};
\draw[color=black] (12,-0.4) node [anchor=north east] {Casas decimais corretas};
\draw[->,color=black] (0,-2) -- (0,10);
\foreach \y in {-1,1,2,3,4,5,6,7,8}
\draw[shift={(0,\y)},color=black] (1pt,0pt) -- (-1pt,0pt) node[left] {\footnotesize $\y$};
\draw[color=black] (-0.7,10) node [anchor=east] [rotate=90] {Tempo($log_{10}(seg)$)};
\draw[color=black] (0pt,-10pt) node[right] {\footnotesize $0$};
\clip(-2,-2) rectangle (12,10);
\begin{scriptsize}
\fill [color=ca] ( 0 , 0 ) circle (1.5pt);
\fill [color=ca] ( 1 , 0 ) circle (1.5pt);
\fill [color=ca] ( 2 , 0 ) circle (1.5pt);
\fill [color=ca] ( 3 , 0 ) circle (1.5pt);
\fill [color=ca] ( 4 , 0 ) circle (1.5pt);
\fill [color=ca] ( 5 , 0 ) circle (1.5pt);
\fill [color=ca] ( 6 , 0.652 ) circle (1.5pt);
\fill [color=ca] ( 7 , 1.664 ) circle (1.5pt);
\fill [color=ca] ( 8 , 2.682 ) circle (1.5pt);
\fill [color=ca] ( 9 , 3.84 ) circle (1.5pt);
\end{scriptsize}
\draw [color=cb] ( 0 , 0 )--( 1 , 0 );
\draw [color=cb] ( 1 , 0 )--( 2 , 0 );
\draw [color=cb] ( 2 , 0 )--( 3 , 0 );
\draw [color=cb] ( 3 , 0 )--( 4 , 0 );
\draw [color=cb] ( 4 , 0 )--( 5 , 0 );
\draw [color=cb] ( 5 , 0 )--( 6 , 0.652 );
\draw [color=cb] ( 6 , 0.652 )--( 7 , 1.664 );
\draw [color=cb] ( 7 , 1.664 )--( 8 , 2.682 );
\draw [color=cb] ( 8 , 2.682 )--( 9 , 3.84 );
\end{tikzpicture}