Leonhard Paul Euler, nasceu na Basileia, Sui�a, em 1707. Sagrou-se como um grandes nomes da matem�tica devido a seus estudos nas �reas de Teoria dos N�meros, An�lise Matem�tica, Geometria, Teoria dos Grafos, entre outras. Euler iniciou sua vida acad�mica na Universidade de Basileia, em 1720, quando ingressou na Faculdade de Filosofia. Graduou-se em Filosofia no ano de 1723 e, em 1727, aceitou o convite para ingressar na Acad�mia de de S�o Petersburgo, para onde voltou em 1766, ap�s passar 25 anos (1741-1766) na Acad�mia Real da Pr�ssia, em Berlim, a convite de Frederico II. Euler faleceu em 1783, em S�o Petersburgo\cite{Santos:Euler}.

Entre as contribui��es de Euler para a matem�tica, est� incluso o estudo de \textit{s�ries infinitas}, onde � feita a soma com um n�mero infinito de parcelas. Dentro desse estudo, Euler provou que, a constante $\pi$ pode ser aproximada das seguintes formas:
\begin{equation}
\frac{\pi ^ 2}{6} = 1 + \frac{1}{2^2} + \frac{1}{3^2} + \frac{1}{4^2} + \frac{1}{5^2} + \ldots \therefore \pi = \sqrt{6\sum_{n = 0}^{\infty}\frac{1}{n^2}}\label{form:Euler_01}
\end{equation}
\input{./files/pi/metodos/grafEuler_01}
\begin{equation}
\frac{\pi ^ 4}{90} = 1 + \frac{1}{2^4} + \frac{1}{3^4} + \frac{1}{4^4} + \frac{1}{5^4} + \ldots \therefore \pi = \sqrt[4]{90\sum_{n = 0}^{\infty}\frac{1}{n^4}}\label{form:Euler_02}
\end{equation}