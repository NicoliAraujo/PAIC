Leonhard Paul Euler, nasceu na Basileia, Sui�a, em 1707. Sagrou-se como um grandes nomes da matem�tica devido a seus estudos nas �reas de Teoria dos N�meros, An�lise Matem�tica, Geometria, Teoria dos Grafos, entre outras. Euler iniciou sua vida acad�mica na Universidade de Basileia, em 1720, quando ingressou na Faculdade de Filosofia. Graduou-se em Filosofia no ano de 1723 e, em 1727, aceitou o convite para ingressar na Acad�mia de de S�o Petersburgo, para onde voltou em 1766, ap�s passar 25 anos (1741-1766) na Acad�mia Real da Pr�ssia, em Berlim, a convite de Frederico II. Euler faleceu em 1783, em S�o Petersburgo\cite{Santos:Euler}.

Entre as contribui��es de Euler para a matem�tica, est� incluso o estudo de \textit{s�ries infinitas}, onde � feita a soma com um n�mero infinito de parcelas. Dentro desse estudo, Euler provou que, a constante $\pi$ pode ser aproximada das seguintes formas:
\begin{equation}
\frac{\pi ^ 2}{6} = 1 + \frac{1}{2^2} + \frac{1}{3^2} + \frac{1}{4^2} + \frac{1}{5^2} + \ldots \therefore \pi = \sqrt{6\sum_{n = 0}^{\infty}\frac{1}{n^2}}\label{form:Euler_01}
\end{equation}
\definecolor{ca2}{rgb}{ 1.0 , 0.0 , 0.0 }
\definecolor{cb2}{rgb}{ 1.0 , 0.0 , 0.0 }
\begin{tikzpicture}[line cap=round,line join=round,>=triangle 45,x=1.0cm,y=1.0cm]
\draw[->,color=black] (-2,0) -- (12,0);
\foreach \x in {-1,1,2,3,4,5,6,7,8,9,10}
\draw[shift={(\x,0)},color=black] (0pt,1pt) -- (0pt,-1pt) node[below] {\footnotesize $\x$};
\draw[color=black] (12,-0.4) node [anchor=north east] {Casas decimais corretas};
\draw[->,color=black] (0,-2) -- (0,10);
\foreach \y in {-1,1,2,3,4,5,6,7,8}
\draw[shift={(0,\y)},color=black] (1pt,0pt) -- (-1pt,0pt) node[left] {\footnotesize $\y$};
\draw[color=black] (-0.7,10) node [anchor=east] [rotate=90] {Tempo($log_{10}(seg)$)};
\draw[color=black] (0pt,-10pt) node[right] {\footnotesize $0$};
\clip(-2,-2) rectangle (12,10);
\begin{scriptsize}
\fill [color=ca2] ( 0 , 0 ) circle (1.5pt);
\fill [color=ca2] ( 1 , 0 ) circle (1.5pt);
\fill [color=ca2] ( 2 , 0 ) circle (1.5pt);
\fill [color=ca2] ( 3 , 0 ) circle (1.5pt);
\fill [color=ca2] ( 4 , 0 ) circle (1.5pt);
\fill [color=ca2] ( 5 , 0 ) circle (1.5pt);
\fill [color=ca2] ( 6 , 0.538 ) circle (1.5pt);
\fill [color=ca2] ( 7 , 1.546 ) circle (1.5pt);
\fill [color=ca2] ( 7 , 2.664 ) circle (1.5pt);
\fill [color=ca2] ( 7 , 3.7 ) circle (1.5pt);
\end{scriptsize}
\draw [color=cb2] ( 0 , 0 )--( 1 , 0 );
\draw [color=cb2] ( 1 , 0 )--( 2 , 0 );
\draw [color=cb2] ( 2 , 0 )--( 3 , 0 );
\draw [color=cb2] ( 3 , 0 )--( 4 , 0 );
\draw [color=cb2] ( 4 , 0 )--( 5 , 0 );
\draw [color=cb2] ( 5 , 0 )--( 6 , 0.538 );
\draw [color=cb2] ( 6 , 0.538 )--( 7 , 1.546 );
\draw [color=cb2] ( 7 , 1.546 )--( 7 , 2.664 );
\draw [color=cb2] ( 7 , 2.664 )--( 7 , 3.7 );
\end{tikzpicture}
\begin{equation}
\frac{\pi ^ 4}{90} = 1 + \frac{1}{2^4} + \frac{1}{3^4} + \frac{1}{4^4} + \frac{1}{5^4} + \ldots \therefore \pi = \sqrt[4]{90\sum_{n = 0}^{\infty}\frac{1}{n^4}}\label{form:Euler_02}
\end{equation}