\documentclass[12pt]{article}
\usepackage{pgf,tikz}
\usetikzlibrary{arrows}
\usepackage{sbc-template}
\usepackage{graphicx,url}
\usepackage{courier}
\usepackage[latin1]{inputenc}
\usepackage[brazil]{babel}
\usepackage{amssymb}
\usepackage{amsmath}
\usepackage{algorithmic}
\usepackage{float, cite}
\usepackage{booktabs}
\usepackage[bf,sf,footnotesize,indent,justification=centering]{caption}
\usepackage{framed}
\usepackage[caption=true,font=footnotesize]{subfig}
\usepackage{relsize}
\newcommand\Cpp{C\nolinebreak[4]\hspace{-.05em}\raisebox{.4ex}{\relsize{-3}{\textbf{++}}}\ }
\captionsetup[subfigure]{justification=centering,labelfont={bf,sf},textfont={bf,sf,footnotesize},singlelinecheck=off,justification=centering}
\captionsetup[figure]{justification=centering,labelfont={bf,sf},textfont={bf,sf,footnotesize},singlelinecheck=off}
\captionsetup[table]{justification=centering,labelfont={bf,sf},textfont={bf,sf,footnotesize},singlelinecheck=off,justification=raggedright}

\newcounter{definitionCounter}
\newenvironment{definition}[1]{
    \refstepcounter{definitionCounter}
\vspace{0.35cm}
\noindent \textbf{Defini��o \thedefinitionCounter} (\textbf{#1}).
}{\vspace{0.35cm}}


\usepackage{color}
\usepackage{listings}
\lstset{language=C++,
                basicstyle=\ttfamily,
                keywordstyle=\color{blue}\ttfamily,
                stringstyle=\color{red}\ttfamily,
                commentstyle=\color{green}\ttfamily,
                morecomment=[l][\color{magenta}]{\#},
                numbers=left,
                stepnumber=1,
                firstnumber=1,
                numberfirstline=true,
}

\usepackage{tikz}
\usetikzlibrary{arrows}





\hyphenation{}


\sloppy

\title{Aproxima��o dos valores de $\pi$\\no Raspberry Pi}

\author{Janderson Lira, Ello� B. Guedes}

\address{N�cleo de Computa��o\\
Escola Superior de Tecnologia\\
Universidade do Estado do Amazonas\\
Av. Darcy Vargas, 1200 -- Manaus -- Amazonas
    \email{jnl.eng@uea.edu.br, ebgcosta@uea.edu.br}
}



\begin{document}
\selectlanguage{brazil}

\maketitle

\section{Hist�ria de $\pi$} \label{sec:histPi}
A constante num�rica $\pi$ (l�-se `pi') � um n�mero irracional correspondente aproximadamente ao valor $3,1415\ldots$. As primeiras evid�ncias da exist�ncia desta constante surgiram h� cerca de $4$ mil anos na Babil�nia e no Egito. O Papiro de Rhind, escrito aproximadamente em 1700 a.C., cont�m a afirma��o de que  ``a �rea de um circulo � igual a �rea de um quadrado cujo lado � o di�metro do circulo diminu�do de sua nona parte.'' Acredita-se que este seja o mais antigo vest�gio escrito de uma estimativa do valor de $\pi$ \cite{Fraga:NumeroPi}. 

Existem muitos m�todos para calcular o valor de $\pi$. O primeiro m�todo, conhecido como \emph{m�todo cl�ssico}, consistia em enrolar uma corda em torno de algum objeto circular, marcando o ponto em que se tinha uma volta completa. Em seguida, media-se quantas vezes esse peda�o de corda era maior que o di�metro da circunfer�ncia do objeto utilizado \cite{Santos:BreveHistoria}

A primeira tentativa cient�fica de aproximar o valor de $\pi$ � atribu�da a Arquimedes de Siracusa, que obteve a inequa��o 
\begin{equation}
\frac{223}{71} < \pi < \frac{22}{7},\label{eq:desigualdade}
\end{equation} obtendo o valor de $\pi$ igual a $3,14$, com duas casas decimais por meio do m�todo cl�ssico utilizando um pol�gono de $96$ lados. Trabalhos posteriores, que datam at� o ano de $1621$, tentaram melhorar esta aproxima��o, mas ainda mantendo a utiliza��o do m�todo cl�ssico \cite{Santos:BreveHistoria}.

Apenas em meados do S�culo XVII surgiram outros m�todos, a exemplo dos m�todos que envolviam a utiliza��o de s�ries infinitas voltadas especialmente para aproxima��o das casas decimais de $\pi$. M�todos que utilizavam rela��es trigonom�tricas tamb�m foram desenvolvidos. A combina��o destes dois m�todos para obten��o de $\pi$ tamb�m foi considerada em estudos posteriores. At� o ano de $1948$, utilizando os m�todos mencionados, obteve-se conhecimento de $808$ casas decimais de $\pi$.

Muitos matem�ticos n�o ocuparam-se apenas em calcular as casas decimais de $\pi$, como tamb�m em provar determinados conceitos matem�ticos sobre este n�mero. O matem�tico Johann Lambert, por exemplo provou que $\pi$ � um n�mero irracional. Anos depois,  Adrien-Merie Legendre provou que $\pi^2$ � irracional e Ferdinand Lindenmann provou que $\pi$ � um n�mero transcendente.
%ESCREVER A CRONOLOGIA DE PI AP�S O ADVENTO DO COMPUTADOOR
%REFAZER O GR�FICO COMPLETO
%\begin{tikzpicture}[x=1cm,y=0.005cm]

  \def\xmin{0}
  \def\xmax{13}
  \def\ymin{0}
  \def\ymax{820}

  % axes
  \draw[->] (\xmin,\ymin) -- (\xmax,\ymin) node[right] {ano};
  \draw[->] (\xmin,\ymin) -- (\xmin,\ymax);

  % xticks and yticks
\node at (1, \ymin) [below] {\tiny{240 a.C.}};
\node at (2, \ymin) [below] {\tiny{480 d.C.}};
\node at (3, \ymin) [below] {\tiny{(\ldots)}};
\node at (4, \ymin) [below] {\tiny{1426}};
\node at (5, \ymin) [below] {\tiny{1579}};
\node at (6, \ymin) [below] {\tiny{1596}};
\node at (7, \ymin) [below] {\tiny{1621}};
\node at (8, \ymin) [below] {\tiny{1699}};
\node at (9, \ymin) [below] {\tiny{1841}};
\node at (10, \ymin) [below] {\tiny{1853}};
\node at (11, \ymin) [below] {\tiny{1873}};
\node at (12, \ymin) [below] {\tiny{1948}};

\node [black,label={[label distance=1cm]},label=\small{$2$}] at (1,2){\textbullet};
\node [black,label={[label distance=1cm]},label=\small{$6$}] at (2,6){\textbullet};
\node [black,label={[label distance=1cm]},label=\small{$16$}] at (4,16){\textbullet};
\node [black,label={[label distance=1cm]},label=\small{$19$}] at (5,19){\textbullet};
\node [black,label={[label distance=1cm]},label=\small{$30$}] at (6,30){\textbullet};
\node [black,label={[label distance=1cm]},label=\small{$39$}] at (7,39){\textbullet};
\node [black,label={[label distance=1cm]},label=\small{$71$}] at (8,71){\textbullet};
\node [black,label={[label distance=1cm]},label=\small{$208$}] at (9,208){\textbullet};
\node [black,label={[label distance=1cm]},label=\small{$400$}] at (10,400){\textbullet};
\node [black,label={[label distance=1cm]},label=\small{$707$}] at (11,707){\textbullet};
\node [black,label={[label distance=1cm]},label=\small{$808$}] at (12,808){\textbullet};
\draw[] (0,0) -- node[pos=0.5,above,rotate=90]{casas decimais} (0,820);
\end{tikzpicture}
\section{O Raspberry Pi B} \label{sec:rasp}
O \textit{Raspberry Pi}(Rasp) � um computador de pequeno porte, com consider�vel poder de processamento e encapsulado em uma �nica placa de circuito impresso. A vers�o utilizado nesse projeto � o \textit{Rasberry Pi B}(RaspB), que possui dimens�es de 85,6x53,98x17mm.

Baseado na arquitetura ARM11, o processador do RaspB � um \textit{System-on-Chip}(SoC) Broadcom BCM2835. Ao BCM2835, podem ser acoplados:
\begin{itemize}
\item{Temporizadores};
\item{Controladores de interrup��es};
\item{\textit{General Purpose Input/Output}(GPIO)};
\item{\textit{Universal Serial Bus}(USB)};
\item{Controlador de \textit{Direct Memory Access}(DMA)};
\item{entre outros}.
\end{itemize}
O RaspB possui uma CPU ARM1176JZFS com clock de 700MHz(isso significa que o processador do RaspB pode realizar 700 milh�es de opera��es por segundo), acompanhado de uma mem�ria principal SDRAM de 512MB. Sua GPU possui um co-processador multim�dia VideoCore IV$^\circledR$ Dual Core e um acelerador gr�fico OpenVG$^\circledR$ Open GL ES 2.0 com performace de at� 24GFLOPS, capas de reproduzir at� 1080p de resolu��o \cite{Holton:RaspArch}.

O boot do sistema operacional � dado atrav�s de cart�o \textit{microSD} inserido no RaspB. Este modelo � apenas compat�vel com distribui��es Linux. A fabricante aconselha que, para melhor utiliza��o dos recursos do RaspB, seja usada a distribui��o \textit{Raspbian Wheezy}  \cite{Upton:UserGuide}.

Para melhor funcionamento do RaspB, � necess�ria uma fonte microUSB que possa fornecer 5V de tens�o e uma corrente entre 700 e 1200mA. Com corrente abaixo dessa faixa, o RaspB n�o funcionar� corretamente e, correntes acima dessa faixa, podem causar s�rios danos ao RaspB. Sobre as conex�es de Entrada/Sa�da(I/O, do ingl�s \emph{Input/Outpu}), o RaspB possui:
\begin{itemize}
\item{Duas portas USB 2.0};
\item{Duas sa�das de v�deo e �udio, sendo uma \emph{High-Definition Multimedia Interface}(HDMI) e outra RCA de 3,5mm};
\item{Interface serial para c�mera  MIPI(CSI-2)};
\item{Interface serial para display(DSI) com conector para cabo \emph{flat}}.
\end{itemize}
\section{Aplica��es do Raspberry Pi}\label{sec:raspApli}
\begin{tabular}{|p{4cm}|p{2.2cm}|p{7.5cm}|}
\hline 
Fonte & �rea & Forma de Aplica��o \\ 
\hline\hline
\scriptsize\url{http://www.scielo.br/pdf/rb/v47n2/pt_0100- 3984-rb-47-02-99.pdf}&
Medicina &
Utilizando softwares livres para visualiza��o de arquivos DICOM, � poss�vel, atrav�s da sa�da de v�deo de alta de defini��o de at� 2,2 megapixels (HDMI, 1920 x 1200 pixels) criar esta��es de visualiza��es de exames, sejam eles oriundo de radiografia, mamografia, ultrassonografia, tomografia computadorizada ou resson�ncia magn�tica. \\
\hline
\scriptsize{\url{https://www.revistas.unijui.edu.br/index.php/salaoconhecimento/article /download/1949/1614}}&
Engenharia El�trica&
Visando a redu��o de custos e a melhoria no desempenho dos atuais sistemas de medi��o de m�dia tens�o, foi contru�do um equipamento, usando o Raspberry Pi como base, capaz de captar, em m�dia, 400 amostras por ciclo. Utilizando um cart�o de mem�ria de 16Gb � poss�vel manter o sistema funcionando por cerca de 24 horas. Como o Raspberry Pi possui os mesmos recursos de um computador convencional, � poss�vel fazer a an�lise minuciosa desses sinais. \\
\hline
\scriptsize{\url{http://cibem7.semur.edu.uy/7/actas/pdfs/143.pdf}}&
Matem�tica&
Com o objetivo de incentivar o estudo da matem�tica atrav�s do meio computacional, o Raspberry Pi, aliado ao software GeoGebra, pode ser utilizado em escolas como um computador de baixo custo capaz de proporcionar uma boa experi�ncia matem�tica aos alunos dos ensinos fundamental e m�dio. \\
\hline
\scriptsize{\url{http://pdf.blucher.com.br/mathematicalproceedings/cnmai2014/0144.pdf}}&
Automa��o Industrial&
O Raspberry Pi pode ser utilizado como controlador de qualquer dispositivo atuador, neste caso, uma garra eletromec�nica � utilizada. O Raspberry Pi est� conectado � garra e, atrav�s da conex�o de internet, um dispositivo m�vel qualquer, conectado na rede wi-fi local, ir� controlar o Raspberry Pi de qualquer ponto da f�brica. Isso permite a redu��o de custo na contru��o dos equipamentos de controle remoto desses atuadores. \\
\hline
\scriptsize{\url{http://www.daveakerman.com/?p=592}}&
Meteorologia &
Acoplando sensores de temperatura, press�o e uma webcam, � poss�vel fazer com o que Raspberry Pi funciona como um equipamente de capta��o de informa��es e imagens para fins meteriol�gicos e de mapeamento. Fazendo as devidas altera��es e colocando-o em um bal�o meteriol�gico, o Raspberry Pi pode enviar as imagens e informa��es dos sensores para uma um computador em tempo real atrav�s de ondas de r�dio.\\
\hline
\end{tabular} 
\section{M�todos Iterativos para C�lculo de $\pi$}\label{sec:MetIt}
\subsection{M�todo de Leibniz}
Nascido na cidade de Leipzig, em 1646, Gottfried Wilhelm von Leibniz era fil�sofo, matem�tico e conselheiro pol�tico, desenvolveu importantes teorias em suas �reas de atua��o e � conhecido, ao lado de Isaac Newton, como um dos pais do c�lculo moderno, devido suas descobertas sobre o c�lculo diferencial. Leibniz faleceu na cidade de Hanover, em 1716 \cite{Look:GottfriedLeibniz}.

Dentre as contribui��es de Leibniz para a �rea da matem�tica, est� a f�rmula para aproxima��o de $\pi$, obtida por volta de 1670. Existe uma grande semelhan�a a f�rmula de Leibniz e a do matem�tico escoc�s James Gregory (1638-1675), e portanto, ficou conhecido como f�rmula de Gregoryl-Leibniz \cite{Wendpap:AbordagemPi}.

\begin{equation}
\pi = 4(\frac{1}{1}-\frac{1}{3}+\frac{1}{5}-\frac{1}{7}+\ldots) =4\sum_{n=0}^{\infty}\frac{(-1)^n}{2n + 1}\label{form:Leibniz}
\end{equation}

\definecolor{ca}{rgb}{ 0.0 , 0.0 , 0.8 }
\definecolor{cb}{rgb}{ 0.0 , 0.0 , 1.0 }
\begin{tikzpicture}[line cap=round,line join=round,>=triangle 45,x=1.0cm,y=1.0cm]
\draw[->,color=black] (-2,0) -- (12,0);
\foreach \x in {-1,1,2,3,4,5,6,7,8,9,10}
\draw[shift={(\x,0)},color=black] (0pt,1pt) -- (0pt,-1pt) node[below] {\footnotesize $\x$};
\draw[color=black] (12,-0.4) node [anchor=north east] {Casas decimais corretas};
\draw[->,color=black] (0,-2) -- (0,10);
\foreach \y in {-1,1,2,3,4,5,6,7,8}
\draw[shift={(0,\y)},color=black] (1pt,0pt) -- (-1pt,0pt) node[left] {\footnotesize $\y$};
\draw[color=black] (-0.7,10) node [anchor=east] [rotate=90] {Tempo($log_{10}(seg)$)};
\draw[color=black] (0pt,-10pt) node[right] {\footnotesize $0$};
\clip(-2,-2) rectangle (12,10);
\begin{scriptsize}
\fill [color=ca] ( 0 , 0 ) circle (1.5pt);
\fill [color=ca] ( 1 , 0 ) circle (1.5pt);
\fill [color=ca] ( 2 , 0 ) circle (1.5pt);
\fill [color=ca] ( 3 , 0 ) circle (1.5pt);
\fill [color=ca] ( 4 , 0 ) circle (1.5pt);
\fill [color=ca] ( 5 , 0 ) circle (1.5pt);
\fill [color=ca] ( 6 , 0.652 ) circle (1.5pt);
\fill [color=ca] ( 7 , 1.664 ) circle (1.5pt);
\fill [color=ca] ( 8 , 2.682 ) circle (1.5pt);
\fill [color=ca] ( 9 , 3.84 ) circle (1.5pt);
\end{scriptsize}
\draw [color=cb] ( 0 , 0 )--( 1 , 0 );
\draw [color=cb] ( 1 , 0 )--( 2 , 0 );
\draw [color=cb] ( 2 , 0 )--( 3 , 0 );
\draw [color=cb] ( 3 , 0 )--( 4 , 0 );
\draw [color=cb] ( 4 , 0 )--( 5 , 0 );
\draw [color=cb] ( 5 , 0 )--( 6 , 0.652 );
\draw [color=cb] ( 6 , 0.652 )--( 7 , 1.664 );
\draw [color=cb] ( 7 , 1.664 )--( 8 , 2.682 );
\draw [color=cb] ( 8 , 2.682 )--( 9 , 3.84 );
\end{tikzpicture}
\subsection{M�todos de Euler}
Leonhard Paul Euler, nasceu na Basileia, Sui�a, em 1707. Sagrou-se como um grandes nomes da matem�tica devido a seus estudos nas �reas de Teoria dos N�meros, An�lise Matem�tica, Geometria, Teoria dos Grafos, entre outras. Euler iniciou sua vida acad�mica na Universidade de Basileia, em 1720, quando ingressou na Faculdade de Filosofia. Graduou-se em Filosofia no ano de 1723 e, em 1727, aceitou o convite para ingressar na Acad�mia de de S�o Petersburgo, para onde voltou em 1766, ap�s passar 25 anos (1741-1766) na Acad�mia Real da Pr�ssia, em Berlim, a convite de Frederico II. Euler faleceu em 1783, em S�o Petersburgo\cite{Santos:Euler}.

Entre as contribui��es de Euler para a matem�tica, est� incluso o estudo de \textit{s�ries infinitas}, onde � feita a soma com um n�mero infinito de parcelas. Dentro desse estudo, Euler provou que, a constante $\pi$ pode ser aproximada das seguintes formas:
\begin{equation}
\frac{\pi ^ 2}{6} = 1 + \frac{1}{2^2} + \frac{1}{3^2} + \frac{1}{4^2} + \frac{1}{5^2} + \ldots \therefore \pi = \sqrt{6\sum_{n = 0}^{\infty}\frac{1}{n^2}}\label{form:Euler_01}
\end{equation}
\definecolor{ca2}{rgb}{ 1.0 , 0.0 , 0.0 }
\definecolor{cb2}{rgb}{ 1.0 , 0.0 , 0.0 }
\begin{tikzpicture}[line cap=round,line join=round,>=triangle 45,x=1.0cm,y=1.0cm]
\draw[->,color=black] (-2,0) -- (12,0);
\foreach \x in {-1,1,2,3,4,5,6,7,8,9,10}
\draw[shift={(\x,0)},color=black] (0pt,1pt) -- (0pt,-1pt) node[below] {\footnotesize $\x$};
\draw[color=black] (12,-0.4) node [anchor=north east] {Casas decimais corretas};
\draw[->,color=black] (0,-2) -- (0,10);
\foreach \y in {-1,1,2,3,4,5,6,7,8}
\draw[shift={(0,\y)},color=black] (1pt,0pt) -- (-1pt,0pt) node[left] {\footnotesize $\y$};
\draw[color=black] (-0.7,10) node [anchor=east] [rotate=90] {Tempo($log_{10}(seg)$)};
\draw[color=black] (0pt,-10pt) node[right] {\footnotesize $0$};
\clip(-2,-2) rectangle (12,10);
\begin{scriptsize}
\fill [color=ca2] ( 0 , 0 ) circle (1.5pt);
\fill [color=ca2] ( 1 , 0 ) circle (1.5pt);
\fill [color=ca2] ( 2 , 0 ) circle (1.5pt);
\fill [color=ca2] ( 3 , 0 ) circle (1.5pt);
\fill [color=ca2] ( 4 , 0 ) circle (1.5pt);
\fill [color=ca2] ( 5 , 0 ) circle (1.5pt);
\fill [color=ca2] ( 6 , 0.538 ) circle (1.5pt);
\fill [color=ca2] ( 7 , 1.546 ) circle (1.5pt);
\fill [color=ca2] ( 7 , 2.664 ) circle (1.5pt);
\fill [color=ca2] ( 7 , 3.7 ) circle (1.5pt);
\end{scriptsize}
\draw [color=cb2] ( 0 , 0 )--( 1 , 0 );
\draw [color=cb2] ( 1 , 0 )--( 2 , 0 );
\draw [color=cb2] ( 2 , 0 )--( 3 , 0 );
\draw [color=cb2] ( 3 , 0 )--( 4 , 0 );
\draw [color=cb2] ( 4 , 0 )--( 5 , 0 );
\draw [color=cb2] ( 5 , 0 )--( 6 , 0.538 );
\draw [color=cb2] ( 6 , 0.538 )--( 7 , 1.546 );
\draw [color=cb2] ( 7 , 1.546 )--( 7 , 2.664 );
\draw [color=cb2] ( 7 , 2.664 )--( 7 , 3.7 );
\end{tikzpicture}
\begin{equation}
\frac{\pi ^ 4}{90} = 1 + \frac{1}{2^4} + \frac{1}{3^4} + \frac{1}{4^4} + \frac{1}{5^4} + \ldots \therefore \pi = \sqrt[4]{90\sum_{n = 0}^{\infty}\frac{1}{n^4}}\label{form:Euler_02}
\end{equation}


\section*{Agradecimentos}
Os autores agradecem o apoio financeiro provido pela Funda��o de Amparo � Pesquisa do Estado do Amazonas (FAPEAM). Janderson Lira � bolsista do Programa de Apoio � Inicia��o Cient�fica da Universidade do Estado do Amazonas e FAPEAM edi��o $2015-2016$.


\bibliographystyle{sbc}
\bibliography{ref}

\end{document}
