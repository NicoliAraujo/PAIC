\documentclass[12pt]{article}
\usepackage{pgf,tikz}
\usetikzlibrary{arrows}
\usepackage{sbc-template}
\usepackage{graphicx,url}
\usepackage{courier}
\usepackage[latin1]{inputenc}
\usepackage[brazil]{babel}
\usepackage{amssymb}
\usepackage{amsmath}
\usepackage{algorithmic}
\usepackage{float, cite}
\usepackage{booktabs}
\usepackage[bf,sf,footnotesize,indent,justification=centering]{caption}
\usepackage{framed}
\usepackage[caption=true,font=footnotesize]{subfig}
\usepackage{relsize}
\newcommand\Cpp{C\nolinebreak[4]\hspace{-.05em}\raisebox{.4ex}{\relsize{-3}{\textbf{++}}}\ }
\captionsetup[subfigure]{justification=centering,labelfont={bf,sf},textfont={bf,sf,footnotesize},singlelinecheck=off,justification=centering}
\captionsetup[figure]{justification=centering,labelfont={bf,sf},textfont={bf,sf,footnotesize},singlelinecheck=off}
\captionsetup[table]{justification=centering,labelfont={bf,sf},textfont={bf,sf,footnotesize},singlelinecheck=off,justification=raggedright}

\newcounter{definitionCounter}
\newenvironment{definition}[1]{
    \refstepcounter{definitionCounter}
\vspace{0.35cm}
\noindent \textbf{Defini��o \thedefinitionCounter} (\textbf{#1}).
}{\vspace{0.35cm}}


\usepackage{color}
\usepackage{listings}
\lstset{language=C++,
                basicstyle=\ttfamily,
                keywordstyle=\color{blue}\ttfamily,
                stringstyle=\color{red}\ttfamily,
                commentstyle=\color{green}\ttfamily,
                morecomment=[l][\color{magenta}]{\#},
                numbers=left,
                stepnumber=1,
                firstnumber=1,
                numberfirstline=true,
}

\usepackage{tikz}
\usetikzlibrary{arrows}





\hyphenation{}


\sloppy

\title{Aproxima��o dos valores de $\pi$\\no Raspberry Pi}

\author{Janderson Lira, Ello� B. Guedes}

\address{N�cleo de Computa��o\\
Escola Superior de Tecnologia\\
Universidade do Estado do Amazonas\\
Av. Darcy Vargas, 1200 -- Manaus -- Amazonas
    \email{jnl.eng@uea.edu.br, ebgcosta@uea.edu.br}
}



\begin{document}
\selectlanguage{brazil}

\maketitle

\section{Hist�ria de $\pi$} \label{sec:histPi}
A constante num�rica $\pi$ (l�-se `pi') � um n�mero irracional correspondente aproximadamente ao valor $3,1415\ldots$. As primeiras evid�ncias da exist�ncia desta constante surgiram h� cerca de $4$ mil anos na Babil�nia e no Egito. O Papiro de Rhind, escrito aproximadamente em 1700 a.C., cont�m a afirma��o de que  ``a �rea de um circulo � igual a �rea de um quadrado cujo lado � o di�metro do circulo diminu�do de sua nona parte.'' Acredita-se que este seja o mais antigo vest�gio escrito de uma estimativa do valor de $\pi$ \cite{Fraga:NumeroPi}. 

Existem muitos m�todos para calcular o valor de $\pi$. O primeiro m�todo, conhecido como \emph{m�todo cl�ssico}, consistia em enrolar uma corda em torno de algum objeto circular, marcando o ponto em que se tinha uma volta completa. Em seguida, media-se quantas vezes esse peda�o de corda era maior que o di�metro da circunfer�ncia do objeto utilizado \cite{Santos:BreveHistoria}

A primeira tentativa cient�fica de aproximar o valor de $\pi$ � atribu�da a Arquimedes de Siracusa, que obteve a inequa��o 
\begin{equation}
\frac{223}{71} < \pi < \frac{22}{7},\label{eq:desigualdade}
\end{equation} obtendo o valor de $\pi$ igual a $3,14$, com duas casas decimais por meio do m�todo cl�ssico utilizando um pol�gono de $96$ lados. Trabalhos posteriores, que datam at� o ano de $1621$, tentaram melhorar esta aproxima��o, mas ainda mantendo a utiliza��o do m�todo cl�ssico \cite{Santos:BreveHistoria}.

Apenas em meados do S�culo XVII surgiram outros m�todos, a exemplo dos m�todos que envolviam a utiliza��o de s�ries infinitas voltadas especialmente para aproxima��o das casas decimais de $\pi$. M�todos que utilizavam rela��es trigonom�tricas tamb�m foram desenvolvidos. A combina��o destes dois m�todos para obten��o de $\pi$ tamb�m foi considerada em estudos posteriores. At� o ano de $1948$, utilizando os m�todos mencionados, obteve-se conhecimento de $808$ casas decimais de $\pi$.

Muitos matem�ticos n�o ocuparam-se apenas em calcular as casas decimais de $\pi$, como tamb�m em provar determinados conceitos matem�ticos sobre este n�mero. O matem�tico Johann Lambert, por exemplo provou que $\pi$ � um n�mero irracional. Anos depois,  Adrien-Merie Legendre provou que $\pi^2$ � irracional e Ferdinand Lindenmann provou que $\pi$ � um n�mero transcendente.
%ESCREVER A CRONOLOGIA DE PI AP�S O ADVENTO DO COMPUTADOOR
%REFAZER O GR�FICO COMPLETO
%\input{./files/pi/grafico}
\section{O Raspberry Pi B} \label{sec:rasp}
O \textit{Raspberry Pi}(Rasp) � um computador de pequeno porte, com consider�vel poder de processamento e encapsulado em uma �nica placa de circuito impresso. A vers�o utilizado nesse projeto � o \textit{Rasberry Pi B}(RaspB), que possui dimens�es de 85,6x53,98x17mm.

Baseado na arquitetura ARM11, o processador do RaspB � um \textit{System-on-Chip}(SoC) Broadcom BCM2835. Ao BCM2835, podem ser acoplados:
\begin{itemize}
\item{Temporizadores};
\item{Controladores de interrup��es};
\item{\textit{General Purpose Input/Output}(GPIO)};
\item{\textit{Universal Serial Bus}(USB)};
\item{Controlador de \textit{Direct Memory Access}(DMA)};
\item{entre outros}.
\end{itemize}
O RaspB possui uma CPU ARM1176JZFS com clock de 700MHz(isso significa que o processador do RaspB pode realizar 700 milh�es de opera��es por segundo), acompanhado de uma mem�ria principal SDRAM de 512MB. Sua GPU possui um co-processador multim�dia VideoCore IV$^\circledR$ Dual Core e um acelerador gr�fico OpenVG$^\circledR$ Open GL ES 2.0 com performace de at� 24GFLOPS, capas de reproduzir at� 1080p de resolu��o \cite{Holton:RaspArch}.

O boot do sistema operacional � dado atrav�s de cart�o \textit{microSD} inserido no RaspB. Este modelo � apenas compat�vel com distribui��es Linux. A fabricante aconselha que, para melhor utiliza��o dos recursos do RaspB, seja usada a distribui��o \textit{Raspbian Wheezy}  \cite{Upton:UserGuide}.

Para melhor funcionamento do RaspB, � necess�ria uma fonte microUSB que possa fornecer 5V de tens�o e uma corrente entre 700 e 1200mA. Com corrente abaixo dessa faixa, o RaspB n�o funcionar� corretamente e, correntes acima dessa faixa, podem causar s�rios danos ao RaspB. Sobre as conex�es de Entrada/Sa�da(I/O, do ingl�s \emph{Input/Outpu}), o RaspB possui:
\begin{itemize}
\item{Duas portas USB 2.0};
\item{Duas sa�das de v�deo e �udio, sendo uma \emph{High-Definition Multimedia Interface}(HDMI) e outra RCA de 3,5mm};
\item{Interface serial para c�mera  MIPI(CSI-2)};
\item{Interface serial para display(DSI) com conector para cabo \emph{flat}}.
\end{itemize}
\section{Aplica��es do Raspberry Pi}\label{sec:raspApli}
\input{./files/rasp/raspApli}
\section{M�todos Iterativos para C�lculo de $\pi$}\label{sec:MetIt}
\input{./files/pi/metodos/metodos}


\section*{Agradecimentos}
Os autores agradecem o apoio financeiro provido pela Funda��o de Amparo � Pesquisa do Estado do Amazonas (FAPEAM). Janderson Lira � bolsista do Programa de Apoio � Inicia��o Cient�fica da Universidade do Estado do Amazonas e FAPEAM edi��o $2015-2016$.


\bibliographystyle{sbc}
\bibliography{ref}

\end{document}
